%%%%%%%%%%%%%%%%%%%%%%%%%%%%%%%%%%%%%%%%%%%%%%%%%%%%%%%%%%%%%%%%%%%%%%%%
%%%%%%%%%%%%%%%%%%%%%% Simple LaTeX CV Template %%%%%%%%%%%%%%%%%%%%%%%%
%%%%%%%%%%%%%%%%%%%%%%%%%%%%%%%%%%%%%%%%%%%%%%%%%%%%%%%%%%%%%%%%%%%%%%%%

%%%%%%%%%%%%%%%%%%%%%%%%%%%%%%%%%%%%%%%%%%%%%%%%%%%%%%%%%%%%%%%%%%%%%%%%
%% NOTE: If you find that it says                                     %%
%%                                                                    %%
%%                           1 of ??                                  %%
%%                                                                    %%
%% at the bottom of your first page, this means that the AUX file     %%
%% was not available when you ran LaTeX on this source. Simply RERUN  %%
%% LaTeX to get the ``??'' replaced with the number of the last page  %%
%% of the document. The AUX file will be generated on the first run   %%
%% of LaTeX and used on the second run to fill in all of the          %%
%% references.                                                        %%
%%%%%%%%%%%%%%%%%%%%%%%%%%%%%%%%%%%%%%%%%%%%%%%%%%%%%%%%%%%%%%%%%%%%%%%%

%%%%%%%%%%%%%%%%%%%%%%%%%%%% Document Setup %%%%%%%%%%%%%%%%%%%%%%%%%%%%

% Don't like 10pt? Try 11pt or 12pt
\documentclass[10pt]{article}

% This is a helpful package that puts math inside length specifications
\usepackage{calc}
\usepackage{pifont}
\usepackage{marvosym}


% Simpler bibsection for CV sections
% (thanks to natbib for inspiration)
\makeatletter
\newlength{\bibhang}
\setlength{\bibhang}{1em}
\newlength{\bibsep}
{\@listi \global\bibsep\itemsep \global\advance\bibsep by\parsep}
\newenvironment{bibsection}%
{\vspace{-\baselineskip}\begin{list}{}{%
			\setlength{\leftmargin}{\bibhang}%
			\setlength{\itemindent}{-\leftmargin}%
			\setlength{\itemsep}{\bibsep}%
			\setlength{\parsep}{\z@}%
			\setlength{\partopsep}{0pt}%
			\setlength{\topsep}{0pt}}}
	{\end{list}\vspace{-.6\baselineskip}}
\makeatother

% Layout: Puts the section titles on left side of page
\reversemarginpar

%
%         PAPER SIZE, PAGE NUMBER, AND DOCUMENT LAYOUT NOTES:
%
% The next \usepackage line changes the layout for CV style section
% headings as marginal notes. It also sets up the paper size as either
% letter or A4. By default, letter was used. If A4 paper is desired,
% comment out the letterpaper lines and uncomment the a4paper lines.
%
% As you can see, the margin widths and section title widths can be
% easily adjusted.
%
% ALSO: Notice that the includefoot option can be commented OUT in order
% to put the PAGE NUMBER *IN* the bottom margin. This will make the
% effective text area larger.
%
% IF YOU WISH TO REMOVE THE ``of LASTPAGE'' next to each page number,
% see the note about the +LP and -LP lines below. Comment out the +LP
% and uncomment the -LP.
%
% IF YOU WISH TO REMOVE PAGE NUMBERS, be sure that the includefoot line
% is uncommented and ALSO uncomment the \pagestyle{empty} a few lines
% below.
%

%% Use these lines for letter-sized paper
%\usepackage[paper=letterpaper,
%           %includefoot, % Uncomment to put page number above margin
%            marginparwidth=0.7in,     % Length of section titles
%            marginparsep=.05in,       % Space between titles and text
%            margin=0.5in,               % 1 inch margins
%            includemp]{geometry}

% Use these lines for A4-sized paper
\usepackage[paper=a4paper,
%includefoot, % Uncomment to put page number above margin
marginparwidth=24mm,    % Length of section titles
marginparsep=1mm,       % Space between titles and text
margin=15mm,              % 25mm margins
includemp]{geometry}

%% More layout: Get rid of indenting throughout entire document
\setlength{\parindent}{0in}

%% This gives us fun enumeration environments. compactitem will be nice.
\usepackage{paralist}
\usepackage[shortlabels]{enumitem}
% \usepackage[misc]{ifsym}
%% Reference the last page in the page number
%
% NOTE: comment the +LP line and uncomment the -LP line to have page
%       numbers without the ``of ##'' last page reference)
%
% NOTE: uncomment the \pagestyle{empty} line to get rid of all page
%       numbers (make sure includefoot is commented out above)
%
\usepackage{fancyhdr,lastpage}
\pagestyle{fancy}
%\pagestyle{empty}      % Uncomment this to get rid of page numbers
\fancyhf{}\renewcommand{\headrulewidth}{0pt}
\fancyfootoffset{\marginparsep+\marginparwidth}
\newlength{\footpageshift}
\setlength{\footpageshift}
{0.1\textwidth+0.1\marginparsep+0.1\marginparwidth-2in}
\lfoot{\hspace{\footpageshift}%
	\parbox{3.5in}{\, \hfill %
		\arabic{page} of \protect\pageref*{LastPage} % +LP
		%                    \arabic{page}                               % -LP
		\hfill \,}}

% Finally, give us PDF bookmarks
\usepackage{color,hyperref}
\definecolor{darkblue}{rgb}{0.0,0.0,0.3}
\hypersetup{colorlinks,breaklinks,
	linkcolor=darkblue,urlcolor=darkblue,
	anchorcolor=darkblue,citecolor=darkblue}

%%%%%%%%%%%%%%%%%%%%%%%% End Document Setup %%%%%%%%%%%%%%%%%%%%%%%%%%%%


%%%%%%%%%%%%%%%%%%%%%%%%%%% Helper Commands %%%%%%%%%%%%%%%%%%%%%%%%%%%%

% The title (name) with a horizontal rule under it
%
% Usage: \makeheading{name}
%
% Place at top of document. It should be the first thing.
\newcommand{\makeheading}[1]%
{\hspace*{-\marginparsep minus \marginparwidth}%
	\begin{minipage}[t]{\textwidth+\marginparwidth+\marginparsep}%
		{\large \bfseries #1}\\[-0.15\baselineskip]%
		\rule{\columnwidth}{1pt}%
\end{minipage}}

% The section headings
%
% Usage: \section{section name}
%
% Follow this section IMMEDIATELY with the first line of the section
% text. Do not put whitespace in between. That is, do this:
%
%       \section{My Information}
%       Here is my information.
%
% and NOT this:
%
%       \section{My Information}
%
%       Here is my information.
%
% Otherwise the top of the section header will not line up with the top
% of the section. Of course, using a single comment character (%) on
% empty lines allows for the function of the first example with the
% readability of the second example.
\renewcommand{\section}[2]%
{\pagebreak[2]\vspace{1\baselineskip}%
	\phantomsection\addcontentsline{toc}{section}{#1}%
	\hspace{0in}%
	\marginpar{
		\raggedright \scshape #1}#2}

% An itemize-style list with lots of space between items
\newenvironment{outerlist}[1][\enskip\textbullet]%
{\begin{itemize}[#1]}{\end{itemize}%
	\vspace{-0.6\baselineskip}}

% An environment IDENTICAL to outerlist that has better pre-list spacing
% when used as the first thing in a \section
\newenvironment{lonelist}[1][\enskip\textbullet]%
{\vspace{-\baselineskip}\begin{list}{#1}{%
			\setlength{\partopsep}{0pt}%
			\setlength{\topsep}{0pt}}}
	{\end{list}\vspace{-.6\baselineskip}}

% An itemize-style list with little space between items
% \newenvironment{innerlist}[1][\enskip\textbullet]%
\newenvironment{innerlist}[1][\enskip$\circ$]%
{\begin{compactitem}[#1]}{\end{compactitem}}

% An environment IDENTICAL to innerlist that has better pre-list spacing
% when used as the first thing in a \section
\newenvironment{loneinnerlist}[1][\enskip\textbullet]%
{\vspace{-\baselineskip}\begin{compactitem}[#1]}
	{\end{compactitem}\vspace{-.6\baselineskip}}

% To add some paragraph space between lines.
% This also tells LaTeX to preferably break a page on one of these gaps
% if there is a needed pagebreak nearby.
\newcommand{\blankline}{\quad\pagebreak[2]}

% Uses hyperref to link DOI
\newcommand\doilink[1]{\href{http://dx.doi.org/#1}{#1}}
\newcommand\doi[1]{doi:\doilink{#1}}


%%%%%%%%%%%%%%%%%%%%%%%% End Helper Commands %%%%%%%%%%%%%%%%%%%%%%%%%%%

%%%%%%%%%%%%%%%%%%%%%%%%% Begin CV Document %%%%%%%%%%%%%%%%%%%%%%%%%%%%

%\hyphenpenalty = 9999
\def\vs{\vspace{-0.1in}}
\begin{document}
	% \makeheading{Curriculum Vitae\\ [0.3cm] TIEP HUU VU\quad~~~~~~\quad\quad\quad\quad\quad\quad\quad\quad\quad\quad\quad\quad\quad\quad{\small Last update: December 17, 2015}}
	\makeheading{Nghia Doan \hfill {\small Last update: October 26, 2021}}
	
	%% =======================================
	\section{Contact Information}	
	\newlength{\rcollength}\setlength{\rcollength}{3 in}
	\vs	
	\begin{tabular}[t]{@{}p{\textwidth-\rcollength}p{\rcollength}}
	\texttt{E-mail:}\href{mailto:nghia.doan@mail.mcgill.ca}{nghia.doan@mail.mcgill.ca}\\
	\texttt{Google Scholar:}\href{https://scholar.google.com/citations?user=rM-FvcgAAAAJ&hl=en}{https://scholar.google.com} \texttt{Linkedin:}\href{https://www.linkedin.com/in/nghiadt05}{https://www.linkedin.com/in/nghiadt05}
	\texttt{Homepage:}\href{https://nghiadt05.github.io}{https://nghiadt05.github.io}
	\end{tabular}

	%% =======================================
	\section{Education}
	\href{https://www.mcgill.ca/}{\textbf{McGill University}}, Montr\'eal, Queb\'ec, Canada \hfill 2017--August 2022 (expected)
	\begin{outerlist}
		\item Ph.D. in Electrical and Computer Engineering, GPA: {\bf 3.4/4.0}.
		\item Supervisor: Professor \href{https://www.mcgill.ca/ece/warren-gross}{Warren Gross}. 
	\end{outerlist}
	
	\vspace{0.1in}
	\href{https://en.snu.ac.kr/}{\textbf{Seoul National University}}, Seoul, South Korea \hfill 2015--2017
	\begin{outerlist}
		\item MSc in Electrical and Computer Engineering, GPA: {\bf 3.9/4.3}.
		\item Supervisor: Professor \href{http://capp.snu.ac.kr/}{Hyuk-Jae Lee}. 
	\end{outerlist}

	\vspace{0.1in}
	\href{https://en.snu.ac.kr/}{\textbf{Posts \& Telecommunications Institute of Technology}}, Hanoi, Vietnam \hfill 2009--2014
	\begin{outerlist}
		\item B.Eng. in Electrical and Computer Engineering, GPA: {\bf 8.7/10.0} (top of class).
	\end{outerlist}

	%% =======================================
	\section{Honors and Awards}
	\vspace{-0.25in}
	\begin{outerlist}
		\item {\textit{McGill Engineering Doctoral Award}} \hfill 2017
		\item {\textit{A-san Foundation Scholarship}} \hfill 2016
		\item {\textit{Samsung Electronics Award}} \hfill 2015
		\item {\textit{Scholarships for Academic Excellent}} \hfill 2009-2014		
	\end{outerlist}
	%% =======================================
	\section{Technical Skills} % (fold)
	\label{sec:technical_ski}
	\vspace{-0.25in}
	\begin{outerlist}
		\item \textbf{\textit{Programming Languages}}: MATLAB, C/C++, Python, VHDL, Assembly.
		\item \textbf{\textit{Softwares and Libraries}}: Visual Studio Code, \LaTeX, PyTorch, TensorFlow, OpenCV.
		\item \textbf{\textit{Design Tools}}: ModelSim, Quartus, Altium Designer, Proteus, MDK-Arm, AVR Studio.
	\end{outerlist}
	%% =========  ==============================
	\section{Research Topics} % (fold)
	\label{sec:research_exper}
	\vspace{-0.25in}
	\begin{outerlist}
		\item {\bf High-Performance Decoding of Short Polar and Reed-Muller Codes with Machine Learning for 5G/6G Standards} \hfill 2017--present\\
		Utilizing state-of-the-art machine learning techniques in the design of low-complexity and high-performance decoding algorithms for short polar and Reed-Muller codes, which are targeted for 5G-and-beyond communications standards.
	\end{outerlist}
	\begin{outerlist}
		\item {\bf Abnormal Pedestrian Detection Using Surveillance Video Data} \hfill 2016--2017\\
		Design an automated deep learning algorithm that can detect abnormal behaviors of pedestrians including entering prohibited area, running, and moving with abnormal directions.
	\end{outerlist}
	\begin{outerlist}
		\item {\bf Hardware-Friendly Encoding for High-Efficiency Video Coding (HEVC)} \hfill 2015--2016\\
		Design a hardware-friendly integer motion estimation engine used in the HEVC standard.
	\end{outerlist}

	\begin{outerlist}
		\item {\bf ARM-based Controlling System for Quadcopter} \hfill 2014--2015\\
		Implement an RF-based signal decoder, motor-driver functions, sensor fusion algorithms, and simple PID controlers for a quadcopter using a customized Arm Cortex-M4 embedded system.
	\end{outerlist}
	
	%% ================== block:  ==========================
	\section{Selected Publications}
	\vspace{-.25in}
	\begin{itemize}
		\item[]\textbf{Book Chapter}
		\begin{enumerate}
			\item W. J. Gross, \underline{N. Doan}, E. N. Mambou, and S. A. Hashemi, "Deep Learning Techniques for Decoding Polar Codes", Wiley, 2019.
		\end{enumerate}
	
		\item[]\textbf{Journal Papers}
		\begin{enumerate}
			\item \underline{N. Doan}, S. A. Hashemi, M. Mondelli, and W. J. Gross, "Decoding Reed-Muller Codes with Successive Factor-Graph Permutations", IEEE Transactions on Communications, \textbf{{submitted}}.
						
			\item \underline{N. Doan}, S. A. Hashemi, and W. J. Gross, "Successive-Cancellation Decoding of Reed-Muller Codes with Fast Hadamard Transform", IEEE Transactions on Vehicular Technologies, \textbf{{submitted}}.
			
			\item  \underline{N. Doan}, S. A. Hashemi, and W. J. Gross, "Fast Successive-Cancellation List Flip Decoding of Polar Codes", IEEE Access, \textbf{{submitted}}.
			
			\item \underline{N. Doan}, S. A. Hashemi, F. Ercan, T. Tonnellier, and W. J. Gross, "Neural Successive-Cancellation Flip Decoding of Polar Codes", {{Journal of Signal Processing Systems}}, 2020.
			
			\item F. Ercan, T. Tonnellier, \underline{N. Doan}, W. J. Gross, "Practical Dynamic SC-Flip Polar Decoders: Algorithm and Implementation", {{IEEE Transactions on Signal Processing, 2020}}.
			
			\item \underline{N. Doan}, T. S. Kim, C. E. Rhee, and H.-J. Lee, "A hardware-oriented concurrent TZ search algorithm for High-Efficiency Video Coding", EURASIP Journal on Advances in Signal Processing, 2017.
			
			\item \underline{N. Doan}, S. Kim, L. C. Vo, and H.-J. Lee, "Anomalous Trajectory Detection in Surveillance Systems Using Pedestrian and Surrounding Information", IEIE Transactions on Smart Processing and Computing, 2016.
		\end{enumerate}
	  
		\item[]\textbf{Conference Papers}
		\begin{enumerate}						
			\item \underline{N. Doan}, S. A. Hashemi, F. Ercan, and W. J. Gross, "Fast SC-Flip Decoding of Polar Codes with Reinforcement Learning", IEEE International Conference on Communications (ICC), Montreal, Canada, 2021.
			
			\item \underline{N. Doan}, S. A. Hashemi, and W. J. Gross, "Decoding of Polar Codes with Reinforcement Learning", IEEE Global Communications Conference (GLOBECOM), Taipei, Taiwan, 2020.			
			
			\item \underline{N. Doan}, S. A. Hashemi, F. Ercan, T. Tonnellier, and W. J. Gross, "Neural Dynamic Successive Cancellation Flip Decoding of Polar Codes", IEEE International Workshop on Signal Processing Systems (SiPS), Nanjing, China, 2019.

			\item \underline{N. Doan}, S. A. Hashemi, E. N. Mambou, T. Tonnellier, and W. J. Gross, "Neural Belief Propagation Decoding of CRC-Polar Concatenated Codes", IEEE International Conference on Communications (ICC), Shanghai, China, 2019.			
			
			\item \underline{N. Doan}, S. A. Hashemi, M. Mondelli, and W. J. Gross, "On the Decoding of Polar Codes on Permuted Factor Graphs", IEEE Global Communications Conference (GLOBECOM), Abu Dhabi, UAE, 2018.
			
			\item \underline{N. Doan}, S. Ali Hashemi and W. J. Gross, "Neural Successive Cancellation Decoding of Polar Codes", IEEE 19th International Workshop on Signal Processing Advances in Wireless Communications (SPAWC), Kalamata, Greece, 2018.
		\end{enumerate}  
	\end{itemize} 
	
	%% ==============================================================
	\section{References}
	\def\halfblankline{\vspace{0.1in}}
	
	\halfblankline
	Prof. \textbf{Warren Gross}
	\begin{innerlist}
		\item[] 
		McGill University, Montr\'eal, Queb\'ec, Canada\\
		{E-mail: warren.gross@mcgill.ca}
	\end{innerlist}	

	\halfblankline
	Dr. \textbf{Seyyed Ali Hashemi}
	\begin{innerlist}
		\item[] 
		Qualcomm, San Diego, California, USA\\
		{E-mail: hashemi@qti.qualcomm.com}
	\end{innerlist}	

	\halfblankline
	Dr. \textbf{Furkan Ercan}
	\begin{innerlist}
		\item[] 
		Octasic, Montr\'eal, Queb\'ec, Canada\\
		{E-mail: furkan.ercan@octasic.com}
	\end{innerlist}	
\end{document}